% Note that if you want something in single space you can go back and
% forth between single space and normal space by the use of \ssp and
% \nsp.  If you want doublespacing you can use \dsp.  \nsp is normally
% 1.5 spacing unless you use the doublespace option (or savepaper
% option)
%
%(FORMAT) Usually you *don't* want to mess with the spacing for your
%(FORMAT) final version.  If you think/know that the thesis template
%(FORMAT) and/or thesis style file is incorrect/incomplete, PLEASE
%(FORMAT) contact the maintainer.  THANK YOU!!!



\chapter{INTRODUCTION}
\label{chap:intro}
% By labeling the chapter, I can refer to it later using the
% label. (\ref{chap:intro}, \pageref{chap:intro}) Latex will take care
% of the numbering.

The first three sections offer an overview of homology, cohomology, and homotopy theories, and may be skipped by a reader familiar with these topics. Section introduces whatever that is a bit more...Then we go into category theory... 

\section{Homology}


\section{Cohomology}
\subsection{Introduction}
\subsection{Cup product and cohomology ring}
\subsection{Hurewicz theorem}
\subsection{Bockstein Homomorphism}
mathematics\footnote{\texttt{http://en.wikibooks.org/wiki/LaTeX}\quad
  provides a wealth of information regarding \LaTeX, check it out.}.

\section{Homotopy}

Bibliographical citations are relatively easy.  Here is one \cite{ART}
and another citation \cite{Lehto:1976} and we cannot forget Milnor
\cite{Milnor:topdiff}.

You type new paragraphs by just leaving an empty line between them.


\section{Interplay between homotopy and cohomology: Eilenberg MacLane spaces and the Brown representability theorem}


\section{Category Theory and the Yoneda Lemma}
\label{sec:GUI}
Cohomology operations are operations between cohomology groups that commute with homomorphisms induced by continuous maps. Thus, they provide us with another means of distinguishing spaces.

\begin{definition}
	[\textit{Cohomology Operation}]
	A \textit{cohomology operation} is a map $\Theta = \Theta_X : H^m(X; G) \to H^n(X; H)$ between cohomology groups, for any space $X$, and fixed $m$, $n$, $G$, $H$, satisfying  the naturality property that for any map $f: X \to Y$ between spaces, the following diagram commutes:
	\[\begin{tikzcd}
	{H^m(Y; G)} && {H^n(Y;H)} \\
	\\
	{H^m(X; G)} && {H^n(X; H)}
	\arrow["{\theta_Y}", from=1-1, to=1-3]
	\arrow["{f^*}", from=1-1, to=3-1]
	\arrow["{\Theta_X}"', from=3-1, to=3-3]
	\arrow["{f^*}", from=1-3, to=3-3]
	\end{tikzcd}\]
	
\end{definition}

\begin{theorem}\label{cohomology-op}
	\cite[Proposition 4L.1]{Hatcher:478079}
	Fix $m$, $n \in \Z$, $G$, $H$ groups. For a space $Z$, there is a bijection between the set of cohomology operations $\Theta : H^m(Z; G) \to H^n(Z; H)$ and $H^n(K(G, m); H)$ given by $\Theta \mapsto \Theta(\iota)$, where $\iota$ is a fundamental class in $H^m(K(G, m); G)$.
\end{theorem}

We will state and prove a more general result in category theory, the \textit{Yoneda Lemma}, and apply this result to prove Theorem \ref{cohomology-op}. We first introduce the more general concept of a natural transformation between contravariant functors.

\begin{definition}
	[\textit{Natural Transformation}] If $F$ and $G$ are contravariant functors between categories $C$ and $D$, a \textit{natural transformation} $\eta : F \to G$ is a transformation such that for all maps $g : X \to Y$ in $C$, the following diagram commutes: 
	
	\[\begin{tikzcd}
	X && FY  && GY \\
	\\
	Y && FX && GX
	\arrow["{\eta_Y}", from=1-3, to=1-5]
	\arrow["Gg", to=3-5, from=1-5]
	\arrow["{\eta_X}", from=3-3, to=3-5]
	\arrow["Fg", to=3-3, from=1-3]
	\arrow["g"', from=1-1, to=3-1]
	\end{tikzcd}\]
	
\end{definition}

\begin{proposition}\label{nat}
	Let C be the category of CW complexes and morphisms homotopy classes of continuous maps. Then cohomology operations are natural transformations from $C$ to $C$.
\end{proposition}

\begin{proof}
	Immediate from definitions.
\end{proof}

We give a contravariant argument of the Yoneda lemma, so as to apply it to cohomology operations directly. The covariant argument is analogous.

\begin{theorem}
	[\textit{Yoneda Lemma}]
	Let $C$ be a category, and $X$ an object of $C$. Let $h^X : C^{\op} \to \Set$ be the contravariant functor $h^X = \Hom(-, X)$. Then for any contravariant set-valued functor $F : C^{\op} \to \Set$, we have a bijection between the natural transformations from $h^X$ to $F$ and $FX \in \Set$, that is,
	\[ FX \simeq \Nat(h^X, F) \]
\end{theorem}

\begin{proof}
	Consider a natural transformation $\eta : h^X \to F$. Then for any object $Y$ in $C$ and a map $g : Y \to X$, the following square commutes, where $(h^X g )(\beta) = \beta \circ g$ for $\beta \in h^X X$, and $\eta_X$, $\eta_Y$ are the components of $\eta$ at $X, Y$ respectively.
	\[\begin{tikzcd}
	Y && h^X X && FX \\
	\\
	X && h^X Y && FY
	\arrow["{\eta_X}", from=1-3, to=1-5]
	\arrow["{Fg}", to=3-5, from=1-5]
	\arrow["{\eta_Y}", from=3-3, to=3-5]
	\arrow["h^X g", to=3-3, from=1-3]
	\arrow["g"', from=1-1, to=3-1]
	\end{tikzcd}\]
	Let $1_X \in h^X X$ be the identity map. Then $\eta_X (1_X) \in FX$ and 
	\[ \eta_Y(h^X g)(1_X) = \eta_Y(g) = Fg(\eta_X(1_X)).\] 
	Thus for every object $Y$, $\eta_Y$ is determined by $\eta_X(1_X)$. We define $\tau : \Nat(h^X, F) \to FX$ by $\tau(\eta) = \eta_X(1_X)$.
	
	Conversely, any $g : Y \to X$, gives rise to $Fg: FX \to FY$. Let $x \in FX$. We wish to define a natural transformation $\lambda(x) : h^X \to F$. We define components map $(\lambda(x))_Y : h^X Y \to FY$ given by
	\[ (\lambda(x))_Y (g) = Fg(x).\]
	Then $\lambda(x)$ is a natural transformation, that is given any $f : Z \to Y$, we claim we have the following commuting:
	\[\begin{tikzcd}
	Z && h^X Y && FY \\
	\\
	Y && h^X Z && FZ
	\arrow["{(\lambda(x))_Y}", from=1-3, to=1-5]
	\arrow["{Ff}", to=3-5, from=1-5]
	\arrow["{(\lambda(x))_Z}", from=3-3, to=3-5]
	\arrow["h^X f", to=3-3, from=1-3]
	\arrow["f"', from=1-1, to=3-1]
	\end{tikzcd}\]
	Indeed, $g : Y \to X$, $g \in h^X Y$ has $Ff(Fg(x)) = F(g \circ f)(x) = F (h^X f)(g)(x)$. Therefore we can define $\lambda : FX \to \Nat(h^X, F)$, $x \mapsto \lambda(x)$.
	
	Finally, we need to show that $\tau$ and $\lambda$ are inverses. For $x \in FX$, we have $\tau(\lambda(x)) = (\lambda(x))_X(1_X) = F(1_X)(x) = 1_{FX}(x)$, so that $\tau\circ\lambda = 1_{FX}$. For $\eta \in \Nat(h^X, F)$, we have $\lambda(\tau(\eta)) = \lambda(\eta_X(1_X))$. Then for any object $Y$ and any $g: Y \to X$, $\lambda(\eta_X(1_X))_Y(g) = Fg(\eta_X(1_X)) = \eta_Y(g)$ by above. Thus $\lambda(\eta_X(1_X)) = \eta$, so that $\lambda\circ\tau = 1_{\Nat(h^X, F)}$.
	
\end{proof}

\begin{corollary}\label{hom}
	Let $C$ be a category, $X$, $Y$ objects in $C$. Then,
	\[ \Hom(X, Y) \simeq \Nat \left( \Hom(-, X), \Hom(-, Y) \right) \]
\end{corollary}

\begin{proof}
	Let $F = h^Y$, and apply Yoneda lemma.
\end{proof}

We are now in a position to prove Theorem \ref{cohomology-op}.

\begin{proof}
	[\textit{Proof of Theorem \ref{cohomology-op}}]
	By CW-approximation, it suffices to prove the statememt for the case of $Z$ a CW-complex. Then we can identify $H^m(Z;G)$ with $[Z, K(G, m)]$ and likewise $H^n(Z; H)$ with $[Z, K(H; n)]$. 
	
	By Corollary \ref{hom}, 
	\[\Hom(K(G, m), K(H, n)) \simeq \Nat\left(\Hom(Z, K(G, m)), \Hom(Z, K(H, n)\right)
	\simeq \Nat(H^m(Z; G), H^n(Z; H)),\]
	but the natural transformations between the cohomology groups are cohomology operations by Proposition \ref{nat}, and $\Hom(K(G, m), K(H, n))$ is $H^n(K(G, m) ; H)$.
	
	Let $K = K(G, m))$. The  map $\tau$ from the proof of the Yoneda lemma sends a cohomology operation $\Theta$ to $\Theta_K(1_K)$, where $1_K$ is the identity map on $K$. Then $\Theta_K(1_K) = \Theta(\iota)$ for $\iota \in H^m(K, G) = H^m(K(G, m); G)$ with $\iota$ a fundamental class since $1_K = 1^*\iota = \iota$.
\end{proof}


\chapter{MISCELLANEOUS COMMANDS: AN INTRODUCTION TO EQUATIONS,
  THEOREMS, FIGURES AND TABLES}

In this chapter we see how equations, theorems, figures and tables are
created, enumerated and referenced.  We also play around with lengths
of chapter and section headings.  For example, this chapter begins
with a long chapter heading that must conform to the thesis manual.
Later on there is a very long section heading.  These examples show
how the SDSU thesis class file automatically handles formatting.



\section{Basic Math}

You can have fun formulas, such as $x= 7 y^x$.  If you want the
equations displayed you can use two dollar signs, \$\$ to enclose the
mathematics, or you can use

\medskip\noindent
\texttt{%
  \hspace*{2em}$\backslash$begin\{equation*\} \\
  \hspace*{3em}math stuff \\
  \hspace*{2em}$\backslash$end\{equation*\}
}\\%
as in

\medskip\noindent\hspace*{2em}\begin{minipage}{4.5in}
\begin{verbatim}
\begin{equation*}
  \int_{\partial \Omega} \omega = \int_{\Omega} d\omega.
\end{equation*}
\end{verbatim}
\end{minipage}

\medskip\noindent
which produces
\begin{equation*}
  \int_{\partial \Omega} \omega = \int_{\Omega} d\omega.
\end{equation*}

There are several other ways to display equations.  The code for this
one (which you can see in \verb+body.tex+) aligns all the equal
signs.
\begin{align}
  (x+2)^3 & = (x+2)(x+2)^2 \\
  &= (x+2)(x^2+4x+ 4) \\
  &= x^3+ 6x^2 + 12x + 8
\end{align}
Notice that this last set of equations is numbered, but the previous
one is not.  The * in the \LaTeX{} code eliminates the numbering.


\section{Equations}

Enumeration of equations, theorems, definitions, tables, is handled
automatically by \LaTeX.  Each of these items may be given a label
using \verb+\label{<labelname>}+.  The item can then be refereed to
by \verb+\ref{<labelname>}+.  Below we demonstrate how
to create and label an equation. Our first is a general differential
equation,
\begin{equation}
  \dot{x} = f(t,x),\qquad x(0)= x_0. \label{de1}
\end{equation}
To see that the numbering is going fine we insert a matrix system as
follows:
\begin{equation}
  \dot{y} =
  \begin{bmatrix}
    a_1 & 0 & \cdots & 0 \\
    0 & a_2 & \cdots & 0 \\
    \vdots & \vdots & \ddots & \vdots \\
    0 & 0 & \cdots & a_n
  \end{bmatrix}
  y.
  \label{de2}
\end{equation}
The numbering is valuable when one wants to refer to the
Equations~(\ref{de1}) and (\ref{de2}). Note that when referring to
Equation~(\ref{de1}) you must capitalize the word equation. Also, when
you enter a specific equation, figure, or table, \emph{e.g.,}
Eqn.~(\ref{de1}), then you should type a $\tilde{\phantom{x}}$ between
the word Eqn., Fig., or Table and its labeling number to prevent
inappropriate division of the label at the end of a line.

To display an equation without numbering, one uses the math
displaystyle mode which works as follows:
\begin{equation*}
  \dot{y} = g(y),
\end{equation*}
which is an autonomous equation in $y$. The $y$ at the end of the last
sentence is in standard math mode. Further information on equations is
provided in Appendix~A.


\section{Theorems, etc.}

The student needs to highlight important results such as theorems,
hypotheses, or definitions. In this section we investigate how
\LaTeX{} handles definitions, theorems, corollaries, etc.
\begin{definition}   
  A linear differential equation is asymptotically stable if and only
  if all eigenvalues, $\lambda$, of the operator matrix have negative
  real part.
\end{definition}
We follow this with a couple of theorems and a corollary.
\begin{theorem}
  If the matrix $A$ in the linear differential equation,
  \begin{equation}
    \dot{y} = Ay, \qquad y(0) = y_0, \label{lde}
  \end{equation}
  is symmetric, then the solution of {\rm (\ref{lde})} is
  non-oscillatory.
\end{theorem}
\begin{corollary}
  If the matrix $A$ in {\rm (\ref{lde})} is symmetric and has negative
  eigenvalues, then the solution is non-oscillatory and asymptotically
  stable.
\end{corollary}

In order to check how the numbering proceeds we insert here another
theorem.
\begin{theorem}
  If the matrix $H$ in the linear differential equation,
  \begin{equation}
    \dot{y} = Hy, \qquad y(0) = y_0, \label{ldeh}
  \end{equation}
  is antisymmetric, then the solution of {\rm (\ref{ldeh})} is oscillatory.
\end{theorem}

The \texttt{thesis.tex} also defines environments for \texttt{lemma}
and \texttt{proposition} though you can add more if you wish.  For
example sometimes it is useful to add an \texttt{example} style
environment.  See the preamble of the document for more information.


\section{Numbering of Theorems, etc...}

Everyone has their own favorite way to number things; by default all
environments of types \{ \textsc{theorem, corollary, definition,
  example, lemma, proposition, remark} \} share the same counter,
which is reset at the start of a new chapter.  If you want to change
this, search for \texttt{``Independent Counters''} in
\texttt{thesis.tex}.

\section{Figures or How to Get into Real Trouble if You Take Advantage
  of What \LaTeX{} Can Do}

This section shows how to display figures and refer to them in the
text.  \LaTeX{} does have the ability to insert postscript files using
the \verb+graphicx+ package.  Make sure to include
\verb+\usepackage{graphicx}+ in your preamble, that is between the
\LaTeX{} commands \verb+\documentclass+ and
\verb+\begin{document}+. \emph{See}
  \verb+http://en.wikibooks.org/wiki/LaTeX/Importing_Graphics+
  \emph{for information about importing graphics into your
    document.}

To insert a figure that is formatted in encapsulated postscript, which
must include a Bounding Box line which is named \texttt{fname.ps} you
do the following:
\\
\texttt{ \hspace*{0.5in} $\backslash$begin\{figure\}[ht]
  \\
  \hspace*{0.75in}
  $\backslash$includegraphics[width=$\backslash$linewidth]\{fname.eps\}
  \\
  \hspace*{0.75in} $\backslash$caption\{Insert a caption
  here. $\backslash$label\{figlabel\} \}
  \\
  \hspace*{0.5in} $\backslash$end\{figure\} }
\\
to produce the figure. The \textbf{[ht]} argument to the figure
command is a \emph{suggestion} to \LaTeX{} to put the figure
\textbf{[h]}ere, or at the \textbf{[t]}op of the page; \textbf{[p]}
for a separate page is also possible.
Avoid\marginpar{\small\textbf{\textit{Style note}}} putting tables and
figures at the \textbf{[b]}ottom of the page as this is frowned upon
by the thesis manual; the
preference\marginpar{\scriptsize\raggedright\textbf{NEVER put anything
    in the margin like this!!!}} is to put tables and figures right
after they are first referenced, \emph{i.e.}\ \textbf{[h]}ere, but
at the \textbf{[t]}op of the following page is acceptable in cases
where it does not fit \textbf{[h]}ere.  You can make the suggestion
stronger by saying \textbf{[h!]}  for ``\textbf{[h]}ere\textbf{!},''
but the internal rules may still override your suggestion.
``\verb+\linewidth+'' above can be replaced by some
number of inches (or other size \LaTeX{} size measure such as
\texttt{pt}, \texttt{em}, or \texttt{ex}).  This will left justify the
figure.  Centering is a little more complicated.  We place everything
in a \texttt{minipage} environment:
\\
\texttt{ \hspace*{0.5in} $\backslash$begin\{figure\}[ht]
  \\
  \hspace*{0.75in} $\backslash$centering
  \\
  \hspace*{0.75in} $\backslash$begin\{minipage\}\{$x$in\}
  \\
  \hspace*{1.0in}
  $\backslash$includegraphics[width=$\backslash$linewidth]\{fname.ps\}
  \\
  \hspace*{1.0in} $\backslash$caption\{Insert a caption
  here. $\backslash$label\{figlabel\} \}
  \\
  \hspace*{0.75in} $\backslash$end\{minipage\}
  \\
  \hspace*{0.5in} $\backslash$end\{figure\} }

To demonstrate how the department would like to see figures in the
thesis the following is provided. If you are examining these files
with \texttt{xdvi}, you will only see a blank spot. However, both
printed and ghostview methods described in the previous chapter will
allow viewing.  Suppose that we create a figure to graph the curve
\begin{equation}
  y=\sin(\omega t), \label{gr1}
\end{equation}
where $\omega$ is the circular frequency.  Figure~\ref{fig1} is a
graph of Equation~(\ref{gr1}), and figure~\ref{fig:graph} is an
illustration of a mapping in the complex plane. The interval of time
viewed is $t \in [-5,5]$. The figure reference should be denoted by
either Fig.~\ref{fig1} or by Figure~\ref{fig1} with specific figures
capitalized as noted here.
\begin{figure}[ht]
  \centering
  \begin{minipage}{0.9\linewidth}
    \includegraphics[width=\linewidth]{Figures/plot2.eps}
    \caption{This is a graph of the above equation, where the circular
      frequency is taken as $\omega = 2$.\label{fig1} Note: \emph{if
        you need to cite a source (of e.g.~a figure) in the caption,
        include the FULL CITATION, e.g.~} [\S 4.10.4 Figures,
      {\sc Montezuma Publishing}, {\em San Diego State
        University Dissertation and Thesis Manual: Policies,
        Procedures and Format}, Spring 2010.] --- The Easiest way to
      achieve this is to first use the
      \textbf{\texttt{$\backslash$cite\{...\}}} command, build the
      document, then replace the cite command with the appropriate
      text copied from the generated \textbf{\texttt{thesis.bib}}
      file.  If the figure caption is the \emph{only} place the source
      is cited, add a \textbf{\texttt{$\backslash$nocite\{...\}}}
      command to ensure that it shows up in the bibliography.}
  \end{minipage}
\end{figure}

\begin{figure}[ht]
  \centering
  \begin{minipage}{4.5in}
    \includegraphics[width=\linewidth]{Figures/mapping.eps}
    \caption{Mapping $f(x)$ from the complex plane to itself. \label{fig:graph}}
  \end{minipage}
\end{figure}


When you have a collection of figures and large figures, you may want
to delay insertion of them until the end of the chapter. At the end of
this chapter we are including a full page figure (Fig.~\ref{fullfig})
to demonstrate this \LaTeX{} command. Note that if you cannot obtain
postscript figures or are having too much trouble using the technique
described above, then you can use the \verb+\vspace+ command to
provide an empty space in the manuscript, then use the old-fashioned
technique of taping in your figure and photocopying it.

% 
% If you have oversize full page figures, the following
% might come in handy
% 
% \figurecoversheet{\label{figure:oversize} Caption goes here}


\section{Tables}

The Department of Mathematical Sciences does not have specific
requirements on the exact layout of a table. However, the tables
should be easily readable and properly labeled according to the
regulations in the SDSU Thesis Manual. In this section we want to
demonstrate how \LaTeX{} handles tables. More complicated examples can
be found in Lamport's book \cite{LAM,LAM2}. We begin with a small table,
given by Table \ref{tab1} which inserts nicely into the text.  Note
that the same centering trick as was employed for figures is done here
and we set the width of the \texttt{minipage} environment to 1.9
inches.
% 

\begin{table}[hbt]
  \centering
  \begin{minipage}{1.9in}
    \caption{A Small Table for Listing Some Parameters Used in Some
      Numerical Procedure\label{tab1}. LONG CAPTION--- The Department
      of Mathematical Sciences does not have specific requirements on
      the exact layout of a table. However, the tables should be
      easily readable and properly labeled according to the
      regulations in the SDSU Thesis Manual.}
    \begin{tabular}{|c||c|c|c|c||}    \hline
      Trial &	a  &  b & c & $\omega$ \\ \hline \hline
      1 & 5 & 10  & 15 & $\pi$ \\ \hline
      2 & 10 & 20  & 15 & $2\pi$ \\ \hline
    \end{tabular}
  \end{minipage}
\end{table}
% 

The\marginpar{\small\textbf{\textit{Style note}}} manual however
allows for the caption to be a little wider if the table is really
small and so we can use a wider \texttt{minipage} and then center the
table inside there.  See for example Table \ref{wtab} where we used
width of 3.5 inches.
% 
\begin{table}[hbt]
  \centering
  \begin{minipage}{3.5in}
    \centering
    \caption{Another Small Table for Listing Some Parameters Used in a
      Numerical Procedure\label{wtab}.}
    \begin{tabular}{|c||c|c|c|c||}    \hline
      Trial &	a  &  b & c & $\omega$ \\ \hline \hline
      1 & 5 & 10  & 15 & $\pi$ \\ \hline
      2 & 10 & 20  & 15 & $2\pi$ \\ \hline
    \end{tabular}
  \end{minipage}
\end{table}

Note that you can use the \texttt{center} environment instead of
\texttt{$\backslash$centering} but that might add a little bit of
unwanted whitespace.  With \texttt{$\backslash$centering} on the other
hand, you might have to put braces around the text you wish to center
and sometimes need to add a \texttt{$\backslash$par}.  If you use it
inside a \texttt{minipage}, \texttt{table} or \texttt{figure}
environment, you don't have to really worry about that.  Note however
that without the use of \texttt{minipage} you cannot center the
caption as it automatically left aligns itself to conform with the
thesis manual.

Tables can also be left aligned see for example Table \ref{ltab}.
Here we don't use the \texttt{minipage} environment, but we must then
add linebreaks so that the table caption does not go wider then the
table itself.  We need to add then two titles, one for the list of
tables and one for the caption here.  The former will not have line
breaks and the latter will.
% 
\begin{table}[htb]
  \caption[
  Another Such Table but Left Aligned]{
    Another Such\\Table but Left Aligned\label{ltab}}
  \begin{tabular}{|c||c|c|c|c||}    \hline
    Trial &	a  &  b & c & $\omega$ \\ \hline \hline
    1 & 5 & 10  & 15 & $\pi$ \\ \hline
    2 & 10 & 20  & 15 & $2\pi$ \\ \hline
  \end{tabular}
\end{table}

Sometimes a table might not fit onto a single page, in this case you
must not use the \texttt{table} environment, but instead the
\texttt{longtable} environment. Do note that \texttt{longtable}
automatically centers so you need not worry about that.  See Table
\ref{totallyrandom} for some absolutely random numbers.  To use
\texttt{longtable} environment you must include the \texttt{longtable}
package in your preamble. \textbf{see the note in \texttt{thesis.tex}
  on how to fix the longtable entries in the ``List of Tables'' if
  they are incorrect.}

\clearpage%
\setlength\LTcapwidth{0.9\linewidth}
\begin{longtable}{|l|l|l|}
  % You may need to modify the \LTcapwidth if the title wraps too
  % early, or if it makes your table too large
  %
  % Issue the command ---
  % \setlength\LTcapwidth{\linewidth}
  % --- prior ro the longtable environment
  %
  \caption{A Table of Some Totally Random Numbers. Often (when there
  are other \texttt{table}\,s defined before the \texttt{longtable}) it is
  necessary to issue the command \texttt{$\backslash$clearpage}
  prior to the longtable environment; otherwise table-numbering and
  page-ordering of table/longtable object may get VERY
  strange.\label{totallyrandom}} \\

  % Here are our column headings
  \hline
  \multicolumn{1}{|l|}{\textbf{First}} &
  \multicolumn{1}{l|}{\textbf{Second}} &
  \multicolumn{1}{l|}{\textbf{Third}} \\
  \hline \hline
  \endfirsthead

  % Here is the caption on other pages
  \caption*{\tablename\ \thetable{} (Continued)} \\
  \hline \multicolumn{1}{|l|}{\textbf{First}} &
  \multicolumn{1}{l|}{\textbf{Second}} &
  \multicolumn{1}{l|}{\textbf{Third}} \\ \hline \hline
  \endhead

  \multicolumn{3}{r}{\textbf{(table continues)}}
  \endfoot

  \hline
  \endlastfoot

  $16883.20050 \times 64.19591$ & $23174^{2905}$ & $(5112,5468,27117)$ \\ \hline
  $7216.3398 \times 12239.16770$ & $19961^{9127}$ & $(16136,21997,26051)$ \\ \hline
  $15977.29588 \times 5732.19698$ & $14995^{26728}$ & $(28634,14278,17183)$ \\ \hline
  $24699.2338 \times 8803.18474$ & $19221^{28853}$ & $(18539,6044,19259)$ \\ \hline
  $21444.11156 \times 24727.15793$ & $18372^{28126}$ & $(28032,2375,15319)$ \\ \hline
  $4391.18511 \times 4548.30442$ & $1720^{1369}$ & $(3406,21419,16364)$ \\ \hline
  $30135.17285 \times 30643.14550$ & $9216^{213}$ & $(23353,27690,19435)$ \\ \hline
  $19438.13461 \times 25479.5929$ & $2137^{3868}$ & $(30657,17930,22240)$ \\ \hline
  $26015.13194 \times 24615.8566$ & $17585^{10358}$ & $(13114,15259,12079)$ \\ \hline
  $14483.18666 \times 730.30848$ & $16033^{18015}$ & $(28723,30583,27231)$ \\ \hline
  $28936.21168 \times 22153.15603$ & $7838^{2847}$ & $(8315,13767,4984)$ \\ \hline$12183.11656 \times 22915.1655$ & $4903^{3341}$ & $(26271,13469,20927)$ \\ \hline
  $3861.26584 \times 3418.15940$ & $8299^{22084}$ & $(16670,6379,5349)$ \\ \hline
  $1917.2334 \times 3164.29148$ & $31271^{24332}$ & $(18534,14106,32170)$ \\ \hline
  $21381.22421 \times 13170.26365$ & $1836^{24826}$ & $(16512,3492,29730)$ \\ \hline
  $19854.29763 \times 10431.8013$ & $856^{4247}$ & $(11431,16797,12547)$ \\ \hline$748.699 \times 18926.6097$ & $2617^{21261}$ & $(9262,31765,19764)$ \\ \hline
  $826.17531 \times 1102.229$ & $6144^{23524}$ & $(13399,32510,25360)$ \\ \hline
  $5457.16254 \times 28852.2419$ & $3340^{25847}$ & $(12851,11353,26704)$ \\ \hline
  $17098.22785 \times 10733.29645$ & $23533^{11432}$ & $(15804,29630,14049)$ \\ \hline
  $4297.6124 \times 13047.24061$ & $6951^{30578}$ & $(25163,7180,3955)$ \\ \hline
  $15919.20579 \times 3697.8512$ & $26036^{19951}$ & $(4596,28456,23292)$ \\ \hline
  $30444.8539 \times 1877.24380$ & $25637^{24662}$ & $(2345,22515,15427)$ \\ \hline
  $13777.5551 \times 12290.27827$ & $9848^{18414}$ & $(8106,1141,25365)$ \\ \hline$5916.26304 \times 32545.9871$ & $9456^{20356}$ & $(13568,17968,13625)$ \\ \hline
  $752.22564 \times 9313.24044$ & $20240^{17852}$ & $(25921,11852,10721)$ \\ \hline
  $17816.14197 \times 468.475$ & $27975^{6019}$ & $(12765,23034,15867)$ \\ \hline
  $31180.31140 \times 17008.23777$ & $4288^{10545}$ & $(23555,14160,20001)$ \\ \hline
  $11143.27728 \times 5201.24768$ & $28480^{27765}$ & $(1313,19756,15238)$ \\ \hline
  $19165.12910 \times 27090.29887$ & $30726^{8520}$ & $(30355,31201,3727)$ \\ \hline
  $3607.11199 \times 26761.19474$ & $9611^{25133}$ & $(3715,620,29421)$ \\ \hline
  $14260.24175 \times 10813.1493$ & $2551^{5774}$ & $(6694,27319,1486)$ \\ \hline
  $1691.28633 \times 21243.16929$ & $15030^{1385}$ & $(11252,12149,32111)$ \\ \hline
  $19772.9737 \times 30544.23499$ & $13344^{8975}$ & $(17492,50,18586)$ \\ \hline
  $9857.3765 \times 19207.6510$ & $18025^{10614}$ & $(17324,19518,13165)$

\end{longtable}




A larger table, given by Table \ref{tab2} and reproduced from another
document, then you may need to allow an entire page for the
table. This is done by typing the command
\verb+\begin{table}[p]+. This test example is included in the
minipage environment to show how a footnote\footnote{We also need to
  see how a regular footnote appears in the text, so one was inserted
  here. Multiple lines are easily handled by \LaTeX.}  can be added to
a table.  Several problems have been noted before on how \LaTeX\
handles the location of the table in the text.
\begin{table}[p]
  \centering
  \begin{minipage}{3.7in}
    \caption{Computations for Products of the \emph{RRN} Genes at Different
      Growth Rates\label{tab2}}
    \begin{tabular}{|c||c|c|c|c|c||}	 \hline
      $\tau$(min)  &  100  &	60 & 40 & 30 & 24 \\ \hline \hline
      $C$ period & 67 & 50  & 45 & 43 & 42 \\ \hline
      $D$ period & 30 & 27  & 25 & 24 & 23 \\ \hline
      $V_0$ & 0.437 & 0.577 & 0.815 & 1.15 & 1.63 \\ \hline
      $\bar c$\footnote{$\times 1000\ {\rm ribosomes}/\mu{\rm m}^3$.}
      & 11.1 & 16.8 & 22.1 & 28.1 & 31.4 \\ \hline
      $\bar c_{85}$\footnote{$\times 1000\ {\rm ribosomes}/\mu{\rm m}^3$,
        representing the average concentration of the product of the
        \emph{rrn} gene located at $85'$.} & 1.73 & 2.68 & 3.65 & 4.81
      & 5.57 \\ \hline 
      $\bar c_{57}$\footnote{$\times 1000\ {\rm ribosomes}/\mu{\rm m}^3$,
        representing the average concentration of the product of the \emph{rrn} gene
        located at $57'$.} & 1.36 & 1.98 & 2.43 & 2.87 & 2.96 \\ \hline
      $\bar c_{85}({\scriptstyle\times 100})/\bar c$\footnote{Percentage of
        $\bar c$ produced by the \emph{rrn} gene located at $85'$.} & 15.6 & 15.9 &
      16.5 & 17.1 & 17.7 \\ \hline
      $\bar c_{57}({\scriptstyle\times 100})/\bar c$\footnote{Percentage of
        $\bar c$ produced by the \emph{rrn} gene located at $57'$.} & 12.3 & 11.8 &
      11.0 & 10.2 & 9.44 \\ \hline
      $\bar c_{85}/\bar c_{57}$ & 1.27 & 1.35 & 1.50 & 1.68 & 1.88 \\ \hline
      $r$\footnote{Initiations/min/gene.} & 3.75 & 10.27
      & 22.56 & 38.42 & 56.98 \\ \hline
      $c_{max}$\footnote{$\times 1000\ {\rm ribosomes}/\mu{\rm m}^3$, representing
        the maximum concentration during the cell cycle.} & 11.28 & 17.04
      & 22.33 & 28.36 & 31.77 \\ \hline
      $c_{max}/c_{min}$\footnote{Ratio of maximum to minimum concentration
        during the cell cycle.} & 1.041 & 1.036 & 1.027 & 1.024 & 1.026 \\ \hline
    \end{tabular}
  \end{minipage}
\end{table}

\begin{figure}[p]
  \centering
  \begin{minipage}{4.5in}
    \includegraphics[width=\linewidth]{Figures/somb.eps}
    \includegraphics[width=\linewidth]{Figures/cos.eps}
    \caption{The top graph is the function $z = \sin(r)/r$, while
      the bottom surface is the function $z = \cos(r)$. \label{fullfig}}
  \end{minipage}
\end{figure}



\section{Potential Pitfalls}

(Oh yeah, there must be text between sectioning commands...)

\subsection{Tables and Figures}

There is a conflict between the \verb+\usepackage{subfig}+,
\verb+\usepackage{caption}+ and the \verb+sdsu-thesis.cls+ class
specification.  The long table captions show up correctly (bold and
left aligned with table).  Use \verb+\usepackage{subfigure}+ instead
and all captions, as well as the list of tables page show up ok.

If you insist on \verb+\usepackage{subfig}+, make sure to
\textbf{first} issue the command
\verb+\usepackage[bf,labelsep=period,textfont=bf]{caption}+ where the
first "\verb+bf+" makes the labels "Figure n" bold;
\verb+labelsep=period+ says "use '.' instead of ':'; and
\verb+textfont=bf+ makes the caption text bold.  This may solve your
subfig problems.


Table captions (``table titles'' \cite{DTM2010spring}) go
ABOVE\marginpar{\small\textbf{\textit{Style note}}} the table, must be
in \emph{headline style} where ``all major words are capitalized,''
and there is no period at the end of the caption; in figure captions
only the first word is capitalized, and there is a period at the
end. --- \textbf{THE STYLE DOES NOT CURRENTLY ENFORCE THIS, \emph{YOU}
  HAVE TO DO IT MANUALLY.}

Charts, graphs, diagrams, maps, photographs, and other graphic
illustrations should all be labeled as \emph{Figures} \cite[\S4.6.9,
and \S4.10.4]{DTM2010spring}.  Figure captions are capitalized
sentence style in the text; therefore, the List of Figures entries
should be in sentence style.

All tables and figures must be referenced in text \emph{prior} to
their appearance. Those references should be by number.



\subsubsection{Centered Tables Figures}
\label{sec::centered:tab:fig}

It is not as simple as adding \verb+\centering+ into the figure or
table environment as that will center the caption on the page rather
then left align it with the left edge of the figure or table.  So the
way to solve this is to figure out the width of the figure or table
and add it in a minipage and center that.  For example if our table is
2 inches wide when typeset, then we could do
\begin{verbatim}
\begin{table}[ht]
  \centering
  \begin{minipage}{2in}
    \caption{Caption goes here}
    ... here is your table ...
  \end{minipage}
\end{table}
\end{verbatim}



\subsection{Margins}

It is believed that the \verb+sdsu-thesis.cls+ template complies with
the SDSU thesis manual: 1.25 inch left, 1 inch top, bottom and right.
But your \emph{printout} may not give the right measurement, if your
printer/printer-driver scales the document.  You may have to turn off
scaling and/or tweak the settings in the \verb+sdsu-thesis.cls+ file.

Someone said: \emph{``Some laser printers don't do the margins correctly, for
example my printer shifts the page a bit.  You can correct this with
the} \verb+\hoffset+ \emph{and} \verb+\voffset+ \emph{lengths as:}\\
\hspace*{2em}\verb+\hoffset -0.0625in+\\
\hspace*{2em}\verb+\voffset  0.15625in+''


\subsection{Bad Pagebreaks}

Sometimes LaTeX does not do exactly what you want with respect to
pagebreaks.  To solve this you can manually add a \verb+\pagebreak+
command where it should break, or you could add
\verb+\enlargethispage{12pt}+ to make a page slightly larger if
needed; though I'm not sure how the thesis reviewer will look on such
transgressions, so do that at own risk.

Bad pagebreaks in the table of contents (or list of tables/figures):
If you get a bad pagebreak in a table of contents you can force a
pagebreak by: \verb+\addtocontents{toc}{\protect\pagebreak}+ you add
this at the point in your document that corresponds to that place in
the table of contents.  For list of tables and list of figures,
replace `toc' in the line above with `lot' or `lof.'


\subsection{Bad Linebreaks}

Bad linebreaks in chapter, section (subsection, etc...), or
table/figure caption titles: This classfile tries to make all titles
conform to the requirements of the thesis manual, but it is possible
that it gets things wrong and you may want to add linebreaks (the
\verb+\\+ command) yourself.  However, the table of contents title
should not have any linebreaks.  The way you do it is to add an
optional argument to \verb+\chapter+, \verb+\section+,
\verb+\caption+ as in:\\[0.5\baselineskip]
\hspace*{2em}\verb+\chapter[Title for Table of Contents]{Title With\\Linebreaks}+\\[0.5\baselineskip]
Note that for \verb+\caption+'s in figures and tables you might have
to do this whenever you have a small figure or table as the
table/figure environment cannot make the caption only as long as the
figure since it doesn't know how large the figure is until it typesets
everything.  See example above and more examples in the long-example
directory.  You can also solve the \verb+\caption+ issue with minipage
in the same way we do centering, see
section~\ref{sec::centered:tab:fig}.



\subsection{Vertical Space}

This classfile tries to make all the vertical space as required, but
sometimes you may need to modify what it does, or you just need to
insert some vertical space.  You use the \verb+\vspace+ and
\verb+\vspace*+ commands (see \LaTeX{} manual).  You can use positive
or negative length there and \verb+\vspace*+ makes sure the space appears
even if there is a pagebreak in between.  For example to add 2 inches
of space you can add \verb+\vspace{2in}+.


\subsection[Non-Bold Math in the TOC: $x=2\pi/e$]{Bold Math in the Thesis:  $\mathbf{x=\pi}$}

Math in section titles need to be \textbf{bold}, but cannot be bold in
the Table of Contents.

\chapter{SECTIONING --- THE MIDDLE} \label{chapter:middle}

Middle chapter.  Here we put the middle things, that is, things that
are in the middle and not in the beginning or in the end.  Here we
also test all the section, subsection, and other headings.

\textbf{Note\marginpar{\small\textbf{\textit{Style note}}} that
  CHAPTER TITLES need to be in ALL CAPS --- YOU have enter the chapter
  titles in ALL CAPS!!!}

\section{A Section}
\label{secone}

Some section text.  Note that there should ALWAYS be some text in
between two sectioning levels; a \verb+\section+ directly followed by
a \verb+\subsection+ will not go through the review.

\subsection{A Subsection With a Very Long Title To See How That Will
  Look When Printed}
\label{lonelysubsection}

Some subsection text.

\subsubsection{A Subsubsection}
\label{lonelysubsubsection}

Some subsubsection text.

\subsubsubsection{A Subsubsubsection}
\label{lonelysubsubsubsection}

Some subsubsubsection text.  If you are using this, you are
\sout{probably} over-organizing things.

\paragraph{A Paragraph.}

Some paragraph text.  You never really get this deep --- don't be
ridiculous.

\subsection{A Friend of \S\ref{lonelysubsection}}

This subsection is here because \S\ref{lonelysubsection} cannot stand
alone; sub$^{n}$-sectioning commands are supposed to \emph{divide} the
text into logical chunks.

\section{A 2nd Section}
\label{sectwo}

Yeah, no (sub)section can stand alone; so if there is a \S\ref{secone}
there must be a \S\ref{sectwo}. \S\ref{lonelysubsection}, needs a
``friend''; and so does \S\ref{lonelysubsubsection}, and
\S\ref{lonelysubsubsubsection}.

\vspace{\baselineskip}
\centerline{\textbf{\large Bending the Rules :: A Single ``Section'' Without a Number}}
\vspace{0.25\baselineskip}

Occasionally, you may ``need'' a  sub$^n$-single section; in that case
you can circumvent the rules by not numbering it, currently the
\verb+\section*{...}+ commands do not produce the desirable results;
the best quick-fix is:\\[0.25\baselineskip]
\hspace*{2em}\verb+[blank line]+\\
\hspace*{2em}\verb+\vspace{\baselineskip}+\\
\hspace*{2em}\verb+\centerline{\textbf{\large A ``Section'' Without a Number}}+\\
\hspace*{2em}\verb+\vspace{0.25\baselineskip}+\\
\hspace*{2em}\verb+[blank line]+\\





\chapter{REFERENCING}

Below a list of references are provided in the acceptable format for
Master's thesis submission. References are to be numbered and should
appear either alphabetically or in the order of appearance in the
text.  (\LaTeX{} does the former for the student.) For students using
\LaTeX{} these are obtained using the plain style with
\textsc{Bib}\TeX. The Department of Mathematics and Statistics will
accept either the plain style or the SIAM style. (For the SIAM style,
it is recommended that you use the included \texttt{siammod.bst}
version.)  There are references for journal \texttt{article}s
\cite{ART}, \texttt{book}s and \texttt{booklet}s \cite{BOK,BKL},
\texttt{inbook}s, \texttt{incollection}s, and \texttt{inproceedings}
\cite{INC,INB,INP}. \emph{Note\marginpar{\small\textbf{\textit{Style
        note}}} that when you have more than one citation in a single
  bracket they must be in increasing numerical order!}  Other sources
may be \texttt{proceedings} \cite{PRO}, technical reports
(\texttt{techreport}) \cite{TEC}, theses (\texttt{mastersthesis}, or
\texttt{PhDthesis}) \cite{MTH}, or \texttt{unpublished} material
\cite{UNP}.  This should provide a fairly comprehensive list for any
material that the student may encounter.  For additional assistance,
see the graduate adviser in your area of concentration.

If you cite a website \cite{Wikipedia} and you cannot find the year on
the website, you should put "n.d." (not dated) at the end. (this is
true for other reference also.) It must also has the word "accessed"
and the month and year you access the website.  You can change how
things with no author(s) are sorted in the bibliography by supplying a
\texttt{key} entry (see \texttt{thbib.bib}), \emph{e.g.} this news
release \cite{EPA-2010-09-07} will be sorted under ``U,'' the leading
letter of the publishing agency (as preferred by the thesis
publisher).

This \cite{PatentExample} is an example of a patent.
\textbf{\textit{Notice:}} how the \texttt{month} and \texttt{year}
fields in \texttt{thbib.bib} have been abused to force the ``correct''
format.

Some articles \cite{ABCD2019} have many authors, pay attention to how
(1) the entry looks in the \texttt{thbib.bib} file, and (2) in the
final bibliography.


\chapter*{UNTOLD SECRETS}

Some departments (\emph{i.e.}\/ the Department of Mathematics and
Statistics) and programs (\emph{i.e.}\/ the Computational Sciences
program) have their own \LaTeX-thesis reviewer(s)\footnote{Peter
  Blomgren \texttt{$\langle$blomgren@sdsu.edu$\rangle$} is one of
  them!}.  This alternative review \textbf{bypasses the Montezuma
  Publishing review}, and tends to be FASTER and more understanding of
\LaTeX quirks.  Please read
\texttt{LaTeX\_Thesis\_Format\_Details\_[UNOFFICIAL].pdf} in the
\texttt{Resources} folder.

This is not really a secret, but calling it a secret makes it more
exciting than section~\ref{sec:LaTeXReview}?

